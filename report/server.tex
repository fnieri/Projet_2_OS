%! TeX program = xelatex
%! TEX TS-program = xelatex
\documentclass[11pt,a4paper]{article}

% Language
\usepackage{polyglossia}
\setdefaultlanguage[frenchpart=false, frenchfootnote=true, frenchitemlabels=true]{french}
\usepackage{numprint}

% Fonts
\usepackage{fontspec}
\setmainfont{Linux Libertine O}
\setsansfont{Linux Biolinum O}
% \setmonofont{Linux Libertine Mono O}

\usepackage{mathtools}
\usepackage{amssymb}
\usepackage{amsfonts}
\usepackage{graphicx}
\usepackage{float}
\usepackage{listings}
\usepackage{hyperref}
\usepackage{attachfile2}

\usepackage{fullpage}
\usepackage[parfill]{parskip}

\usepackage{xcolor}

%-----------------------------------------------------------

\newcommand{\up}[1]{\textsuperscript{#1}}
\newcommand{\down}[1]{\textsubscript{#1}}

\newcommand{\addcode}[3]{
    \begin{figure}[H]
        \centering
        \lstinputlisting[language=#2, caption=\textattachfile{#1}{#1}, label=#3]{#1}
    \end{figure}
}

\newcommand{\addimg}[4]{
    \begin{figure}[H]
        \centering
        \includegraphics[#2]{#1}
    \caption{#3}
    \label{#4}
    \end{figure}
}

\newcommand{\p}[1]{(p #1)}
\newcommand{\e}[2]{#1 \p{#2}}

%-----------------------------------------------------------

\definecolor{base}{RGB}{250,245,237}
\definecolor{subtle}{RGB}{110,107,135}
\definecolor{love}{RGB}{181,99,122}
\definecolor{gold}{RGB}{235,158,51}
\definecolor{rose}{RGB}{214,130,125}
\definecolor{pine}{RGB}{41,105,130}
\definecolor{foam}{RGB}{87,148,158}
\definecolor{iris}{RGB}{143,122,168}

\hypersetup{
    colorlinks=true,
        allcolors=iris
}
\attachfilesetup{color=iris, author=Boris Petrov}

%-----------------------------------------------------------

\lstset{
    % backgroundcolor=\color{white},        % background color
        basicstyle=\ttfamily,                   % regular style
        breakatwhitespace=true,                 % sets if automatic breaks should only happen at whitespace
        breaklines=true,                        % sets automatic line breaking
        captionpos=t,                           % caption-position
        commentstyle=\itshape\color{subtle},    % comment style
        % deletekeywords={...},                 % delete keywords from the given language
        escapeinside={\%*}{*)},                 % if you want to add LaTeX within your code
        firstnumber=1,                          % start line enumeration with line 1
        frame=tb,                               % adds a frame around the code
        keepspaces=true,                        % keeps spaces in text
        keywordstyle=\color{pine},              % keyword style
        language=C++,                           % language of the code
        % morekeywords={*,...},                 % add more keywords to the set
        numbers=left,                           % position of line-numbers; possible values are (none, left, right)
        numbersep=15pt,                         % distance between line-numbers and the code
        numberstyle=\scriptsize\ttfamily\color{subtle}, % style used for line-numbers
        % rulecolor=\color{black},              % if not set, the frame-color may be changed on line-breaks within not-black text (e.g. comments (green here))
        showspaces=false,                       % show spaces everywhere
        showstringspaces=false,                 % underline spaces within strings only
        showtabs=false,                         % show tabs within strings
        stepnumber=1,                           % step between two line-numbers
        stringstyle=\color{love},               % string literal style
        tabsize=4,	                          % default tabsize
        % title=\lstname                        % show the filename
}

%-----------------------------------------------------------

\title{Concurrencer les géants}
\author{Boris Petrov}
\date{}

\begin{document}
\maketitle
\tableofcontents

%-----------------------------------------------------------

\section{Introduction}

Depuis maintenant plusieurs années, les applications de
communication \emph{via} Internet se sont intégrées dans notre
quotidien. Que ce soit dans le milieu professionnel ou non,
utilisées avec de bonnes intentions ou non, elles sont
partout et jusqu'à récemment leur fonctionnement était pour
nous une boîte noire. C'est ce que nous avons voulu changer
avec ce projet de \emph{ChatBox}.

Ce que nous avons réalisé n'est que la première
étape, mais la plus importante.

\section{Côté serveur}

La partie centrale de tout service de communication est
évidemment celle qui relaie les communications d'un individu
à l'autre. Nous allons l'appeler \emph{serveur}.

\subsection{Format des messages}

Pour notre application, les communications se limitent aux
messages écrits. Ainsi chaque message se compose de l'heure
d'envoi, de son auteur et d'un texte.
Il était donc logique de tout mettre dans une
\verb@struct@, ce que nous avons fait à l'exception
de l'auteur.

En effet, le serveur sait de quel client vient le message et
donc connait le pseudonyme utilisé par ce dernier (fourni
lors de la connexion). Nous avons donc pas besoin de
communiquer l'auteur à chaque envoi depuis un client.

D'un autre côté, cela complique le programme dans le sens où
deux fonctions différentes doivent être faites, une pour
envoyer le message sans l'auteur depuis le client et une
pour relayer le message avec l'auteur depuis le serveur.
Mais nous avons tout de même décidé de suivre cette voie.

\subsection{Structure du serveur}

Pour ce qui est du fonctionnement du serveur,
nous avons décidé de diviser l'exécutions en deux phases
principales, l'initialisation et le \emph{main loop}.

La première permet la création de l'entité serveur. Cela
inclut l'ouverture d'un socket, la liaison de ce dernier à
un port du système et la création de la liste de clients et du
compteur de clients. Cette phase se termine par la mise en
écoute dudit port par le serveur.

Vient ensuite la deuxième phase, également divisée en deux
sous-parties. Nous avons d'abord celle qui détermine le
client qui fait une requête et ensuite celle qui exécute la requête
en tant que telle. Les requêtes peuvent être de deux types :
des demandes de connexion/déconnexion et les envois des
messages.

La section déterminant le client ayant fait une requête est
réalisée grâce à l'appel système bloquant \verb@select@ qui permet de
mettre en écoute plusieurs descripteurs de fichiers (dans
notre cas ceux des clients) et de reprendre l'exécution dès
qu'il y a une activité provenant d'un de ces derniers.

Lorsque l'activité provient d'un client, ce ne peut être
qu'un envoi de message (ou une demande de déconnexion, qui
est en réalité un envoi d'un message vide). Dans ce cas, le
message est relayé à tous les clients.

Si par contre l'activité provient du descripteur du serveur,
c'est que nous nous trouvons face à une demande de connexion de la
part d'un client. Il est ajouté et l'écoute continue.

\subsection{Difficultés}

Nous avons eu des problèmes avec l'envoi des chaînes de
caractères à cause de l'omission du \verb@\0@ terminal.
L'absence de ce dernier caractère résultait en des envois de strings
qui n'était pas lisibles à l'arrivée puisque non délimités.




\section{Côté client}



\end{document}
