\section{Côté client}

La partie client combine ce qui est connexion au serveur,
envoi et réception d'un message et interface graphique,
réalisée en utilisant \emph{ncurses}.

\subsection{Structure du client}

Le client, tout comme le serveur, divise son exécution
en deux phases, l'initialisation et le \emph{main loop}.

\subsection{Initialisation}

Lorsque l'utilisateur lance le programme via la commande
\begin{center}
    \texttt{\$ ./client <pseudo> <ip\_serveur> <port>}
\end{center}

le programme établit une
connexion avec le serveur.
Nous créons un socket de type \verb@SOCK_STREAM@ pour pouvoir communiquer entre
plusieurs utilisateurs, et connectons l'utilisateur au serveur caractérisé
par l'IP et par le port introduits.
Le premier message est envoyé automatiquement et contient l'username.
Si la connexion échoue, le programme s'arrête avec un message d'erreur.
Le socket créé pendant la connexion sera réutilisé pour envoyer et recevoir les messages.
\subsection{Main loop}
L'envoi et la réception des messages s'effectue en concurrence dans le \emph{main loop},
grâce à l'utilisation des threads et l'exclusion mutuelle avec les mutex.
\subsubsection{Envoi des messages}
L'envoi des messages se fait à travers une fonction auxiliaire
\texttt{ui\_get\_input} et la fonction qui envoie les messages
au serveur \texttt{read\_stdin}.

\texttt{read\_stdin} appelle la fonction auxiliaire,
qui lit l'input du client dans l'interface
caractère par caractère. Quand l'utilisateur souhaite
envoyer le message en appuyant sur la touche \verb@Entrée@, l'input est analysé
pour en trouver la longueur, nous prenons le timestamp et
construisons le \verb@struct@ \textbf{message} (cfr. section 2.1).

Ensuite, la structure est envoyée au serveur, qui relaie le message contenu
aux autres utilisateurs (cfr. section 2.3) , il est important que
l'envoi d'un message ne soit pas perturbé par la réception d'un autre en même temps.
Pour éviter cela, le mutex assure que l'unique traitement effectué à ce moment
dans le programme soit l'envoi du message.

\subsubsection{Réception du message}
Comme l'envoi, la réception se divise en deux parties. Une reliée à
l'interface et une reliée à la vérification avec le serveur.

La réception du message dans la fonction \texttt{receive\_other\_users\_messages}
consiste à vérifier si des bytes ont été reçu sur le socket,
si c'est bien le cas, le message est interprété par le client et est ensuite
affiché sur l'interface avec la fonction \texttt{ui\_print\_message} selon le schéma suivant: \\
\begin{center}
    \textit{HH:MM:SS [\textbf{username}] message}
\end{center}
Comme pour l'envoi des messages, il est aussi important que
la réception d'un \textbf{message}, et non pas la vérification de la réception de \textbf{bytes},
soit la seule opération en cours de traitement. Si le "lock" d'un mutex
est effectué au niveau de la réception de bytes, l'utilisateur ne pourrait jamais
envoyer des messages.

\subsection{Traitement des signaux}

La terminaison du programme client peut être causée par deux
évènements. Soit la connexion avec le serveur a été
interrompue, soit l'utilisateur a fourni le caractère EOT
(\verb@Ctrl-D@) avec un champs de message vide
(ce qui équivaut dans notre programme à un EOF).

Lors du premier cas, cela va être vu par le client
lorsqu'il va essayer d'envoyer un message
et se rendre compte que le socket de lecture du serveur a
été fermé. Cette écriture sans succès est généralement
accompagnée par le signal \verb@SIGPIPE@ qui normalement
devrait mettre fin à notre programme.

Cependant, nous
voulons d'abord mettre fin à \emph{ncurses} et afficher un
message utile à l'utilisateur. C'est pourquoi nous avons
décidé d'ignorer ce signal et agir en conséquence grâce à la
variable \verb@errno@ qui contient la valeur \verb@EPIPE@
lors d'une écriture infructueuse.


\section{Difficultés}

Nous avons eu des problèmes avec l'envoi des chaînes de
caractères à cause de l'omission du \verb@\0@ terminal.
L'absence de ce dernier caractère résultait en des envois de strings
qui n'était pas lisibles à l'arrivée puisque non délimités.

Un autre problème que nous avons eu est l'utilisation de
mutexes pour l'interface graphique. En effet, la réception
de caractères et la mise à jour de l'interface doivent être
deux actions séparées. Cependant, en utilisant des mutexes,
la séparation fonctionnait mais la mise à jour n'était pas
instantanée. C'est pourquoi nous avons décidé de ne pas les
utiliser pour l'interface, préférant une mise à jour (de
l'écran) depuis la demande d'input. Ce n'est pas idéal mais
dans notre cas, la solution la plus fluide.

\section{Limitations}
Si à l'envoi d'un nouveau message, d'anciens messages sont effacés de l'écran,
il ne sera plus possible de les lire pour des limitations dues au fait que tous les messages
devraient être stockés sur un fichier à part.

\section{Exemple d'execution}
\addimg{img/exemple.png}{width=\textwidth}{Ceci est un exemple d'execution}{exemple}
