\section{Introduction}

Depuis maintenant plusieurs années, les applications de
communication \emph{via} Internet se sont intégrées dans notre
quotidien. Que ce soit dans le milieu professionnel ou non,
utilisées avec de bonnes intentions ou non, elles sont
partout et jusqu'à récemment leur fonctionnement était pour
nous une boîte noire. C'est ce que nous avons voulu changer
avec ce projet de \emph{ChatBox}.

Ce que nous avons réalisé n'est qu'une étape parmi tant d'autres,
mais la plus fondamentale.

\section{Côté serveur}

La partie centrale de tout service de communication est
évidemment celle qui relaie les communications d'un individu
à l'autre. Nous allons l'appeler \emph{serveur}.

\subsection{Format des messages}

Pour notre application, les communications se limitent aux
messages écrits. Ainsi chaque message se compose de l'heure
d'envoi, de son auteur et d'un texte.
Il était donc logique de tout mettre dans une
\verb@struct@, ce que nous avons fait à l'exception
de l'auteur.

En effet, le serveur sait de quel client vient le message et
donc connait le pseudonyme utilisé par ce dernier (fourni
lors de la connexion). Nous n'avons donc pas besoin de
communiquer l'auteur à chaque envoi depuis un client.

D'un autre côté, cela complique le programme dans le sens où
deux fonctions différentes doivent être faites. Une pour
envoyer le message sans l'auteur depuis le client et une
pour relayer le message avec l'auteur depuis le serveur.
Mais nous avons tout de même décidé de suivre cette voie
pour des raisons de performance (même si la différence
ne doit être que minime).

\subsection{Structure du serveur}

Pour ce qui est du fonctionnement du serveur,
nous avons décidé de diviser l'exécutions en deux phases
principales, l'initialisation et le \emph{main loop}.

La première permet la création de l'entité serveur. Cela
inclut l'ouverture d'un socket, la liaison de ce dernier à
un port du système et la création de la liste de clients et du
compteur de clients. Cette phase se termine par la mise en
écoute dudit port par le serveur.

Vient ensuite la deuxième phase, également divisée en deux
sous-parties. Nous avons d'abord celle qui détermine le
client qui fait une requête et ensuite celle qui exécute la requête
en tant que telle. Les requêtes peuvent être de deux types :
les demandes de connexion/déconnexion et les envois des
messages.

La section déterminant le client ayant fait une requête est
réalisée grâce à l'appel système bloquant \verb@select@ qui permet de
mettre en écoute plusieurs descripteurs de fichiers (dans
notre cas ceux des clients) et de reprendre l'exécution dès
qu'il y a une activité provenant d'un de ces derniers.

Lorsque l'activité provient d'un client, ce ne peut être
qu'un envoi de message (ou une demande de déconnexion, qui
est en réalité un envoi d'un message vide). Dans ce cas, le
message est relayé à tous les clients.

Si par contre l'activité provient du descripteur du serveur,
c'est que nous nous trouvons face à une demande de connexion de la
part d'un client. Il est ajouté et l'écoute continue.

Dans les situations de connexion et de déconnexion d'un
client, le serveur en informe les autres participants participants
grâce à un message ayant comme auteur \verb@Serveur@.

\subsection{Traitement des signaux}

La terminaison du serveur a lieu lors d'une réception du
signal \verb@SIGINT@. Cette dernière quitte simplement le
programme avec un message affiché à l'écran.
